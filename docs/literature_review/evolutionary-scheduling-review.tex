\documentclass[a4paper,12pt]{article}
\usepackage[utf8]{inputenc}
\usepackage{graphicx}
\usepackage{amsmath}
\usepackage{booktabs}
\usepackage{hyperref}
\usepackage{enumitem}
\setlist{nosep}
\setlength{\emergencystretch}{3em} % prevents overfull boxes gracefully

\title{Literature Review: Evolutionary Scheduling of Courses or Work Shifts}
\author{Sabína Ságová\\ Charles University}
\date{27 October 2025}

\begin{document}

\maketitle

\begin{abstract}
Write a short summary (150–200 words) describing the focus of your literature review — evolutionary approaches to scheduling (courses or work shifts), time frame (from 2021 onward), and main themes.
\end{abstract}

\section{Introduction}
Introduce the general topic of scheduling problems (courses, shifts), motivation, why evolutionary algorithms are suitable, and what this review aims to achieve.

\section{Older Trends in Scheduling Research}

\subsection{Older Publications}
Here are details of older key papers:
\begin{enumerate}[leftmargin=2em, labelwidth=1em, labelsep=0.5em, itemsep=1ex]
  \item Application of Evolutionary Algorithms in Project Management \\
    \textbf{Authors/Year:} Christos Kyriklidis and Georgios Dounias (2014) \\  
    \textbf{Problem Type:} Resource Leveling Problem\\
    (Time-Constrained Project Scheduling) \\  
    \textbf{Dataset / Instances:} Small and medium benchmark projects\\
    from public project datasets (e.g., PSPLIB) \\  
    \textbf{Evaluation Method / Metrics:} Objectives include maximum resource usage (Gf),\\
    resource leveling index (RLI), and standard deviation (StD); 50 runs\\
    with statistical evaluation of near-optimality \\  
    \textbf{Algorithm Type:} Genetic Algorithm (GA) \\  
    \textbf{Encoding / Individual Design:} Chromosome encodes start-times\\
    of non-critical activities; critical ones fixed \\  
    \textbf{Operators:} Two-point crossover (70\%), mutation (20\%), elitism (10\%);\\
    local search around elite chromosomes to avoid premature convergence \\  
    \textbf{Comparison Methods:} Exhaustive enumeration for small problems;\\
    compared with heuristic, ACO, ANN, and PSO methods \\  
    \textbf{Key Findings:} GA efficiently finds near-optimal resource profiles,\\
    outperforming traditional methods and scaling well for large projects \\  
    \textbf{Citation Count (to date):} 9 (Google Scholar, 2025) \\[2ex]
\end{enumerate}

\section{Recent Trends in Scheduling Research (2021 – Now)}
Here are details of key papers from 2021:
\begin{enumerate}[leftmargin=2em, labelwidth=1em, labelsep=0.5em, itemsep=1ex]
  \item Paper 2 \\
    \textbf{Authors/Year:} \\  
    \textbf{Problem Type:} \\  
    \textbf{Dataset / Instances:} \\  
    \textbf{Evaluation Method / Metrics:} \\  
    \textbf{Algorithm Type:} \\  
    \textbf{Encoding / Individual Design:} \\  
    \textbf{Operators:} \\  
    \textbf{Comparison Methods:} \\  
    \textbf{Key Findings:} \\  
    \textbf{Citation Count (to date):} \\[2ex]
\end{enumerate}

% ... (rest of your sections stay unchanged)

\section*{References}
\bibliographystyle{plain}
\bibliography{yourbibfile}

\end{document}
